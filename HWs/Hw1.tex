

% Short Sectioned Assignment
% LaTeX Template
% Version 1.0 (5/5/12)
%
% This template has been downloaded from:
% http://www.LaTeXTemplates.com
%`
% Original author:
% Frits Wenneker (http://www.howtotex.com)
%
% License:
% CC BY-NC-SA 3.0 (http://creativecommons.org/licenses/by-nc-sa/3.0/)
%
%%%%%%%%%%%%%%%%%%%%%%%%%%%%%%%%%%%%%%%%%

%----------------------------------------------------------------------------------------
%	PACKAGES AND OTHER DOCUMENT CONFIGURATIONS
%----------------------------------------------------------------------------------------
 
\documentclass[paper=letter, fontsize=11pt]{scrartcl} % A4 paper and 11pt font size
\usepackage{tcolorbox}
\usepackage[T1]{fontenc} % Use 8-bit encoding that has 256 glyphs
\usepackage{fourier} % Use the Adobe Utopia font for the document - comment this line to return to the LaTeX default
\usepackage[english]{babel} % English language/hyphenation
\usepackage{amsmath,amsfonts,amsthm} % Math packages

\usepackage{lipsum} % Used for inserting dummy 'Lorem ipsum' text into the template

\usepackage{sectsty} % Allows customizing section commands
%\allsectionsfont{\centering \normalfont\scshape} % Make all sections centered, the default font and small caps

\usepackage{fancyhdr} % Custom headers and footers
\pagestyle{fancyplain} % Makes all pages in the document conform t  o the custom headers and footers
\fancyhead{} % No page header - if you want one, create it in the same way as the footers below
\fancyfoot[L]{} % Empty left footer
\fancyfoot[C]{} % Empty center footer
\fancyfoot[R]{\thepage} % Page numbering for right footer
\renewcommand{\headrulewidth}{0pt} % Remove header underline
\renewcommand{\footrulewidth}{0pt} % Remove footer underlines
\setlength{\headheight}{13.6pt} % Customize the height of the header

%\numberwithin{equation}{section} % Number equations within sections (i.e. 1.1, 1.2, 2.1, 2.2 instead of 1, 2, 3, 4)
%\numberwithin{figure}{section} % Number figures within sections (i.e. 1.1, 1.2, 2.1, 2.2 instead of 1, 2, 3, 4)
%\numberwithin{table}{section} % Number tables within sections (i.e. 1.1, 1.2, 2.1, 2.2 instead of 1, 2, 3, 4)

%$\setlength\parindent{0pt} % Removes all indentation from paragraphs - comment this line for an assignment with lots of text

%----------------------------------------------------------------------------------------
%	TITLE SECTION
%----------------------------------------------------------------------------------------

\newcommand{\horrule}[1]{\rule{\linewidth}{#1}} % Create horizontal rule command with 1 argument of height

\title{	
\normalfont \normalsize 
\textsc{Math 210A Fall 2018} \\ [25pt] % Your university, school and/or department name(s)
\horrule{0.5pt} \\[0.4cm] % Thin top horizontal rule
\huge Homework 1\\ % The assignment title
\horrule{2pt} \\[0.5cm] % Thick bottom horizontal rule
}

\author{Seyedkamyar Kazemi} % Your name

\date{\normalsize\today} % Today's date or a custom date
\begin{document}

\maketitle
\section{Non-book problems}
\subsection{\textbf{prove that:}}

\subsubsubsection{$(A_1 \cup A_2)\Delta(B_1 \cup B_2) \subset (A_1 \Delta B_1)\cup(A_2 \Delta B_2)$}

    $$(A_1 \cup A_2)\Delta(B_1 \cup B_2)=$$
    $$(A_1 \cap B_1^c\cap B_2^c)\cup (A_2 \cap B_1 ^ c \cap B_2 ^c)\cup(B_1 \cap A_1^c\cap A_2^c)\cup (B_2 \cap A_1 ^ c \cap A_2 ^c)$$


But if 
$$ 1) x \in (A_1 \cap B_1^c\cap B_2^c) \text{ then } x \in (A_1 \ B_1)$$
$$ 2) x \in (A_2 \cap B_1^c\cap B_2^c) \text{ then } x \in (A_2 \ B_2)$$
$$ 3) x \in (B_1 \cap A_1^c\cap A_2^c) \text{ then } x \in (B_1 \ A_1)$$
$$ 4) x \in (B_2 \cap A_1^c\cap A_2^c) \text{ then } x \in (B_2 \ A_2)$$
\subsubsection{$(A_1\cap A_2) \Delta (B_1 \cap B_2) \subset (A_1\Delta A_2) \cup (B_1 \Delta B_2)$}
if $x \in (A_1 \cap A_2) \backslash (B_1 \cap B_2)$ the $x \in  (A_1 \backslash B_1) \cup (A_2 \backslash B_2)$
if $x \in (B_1 \cap B_2) \backslash (A_1 \cap A_2)$ the $x \in  (B_1 \backslash A_1) \cup (B_2 \backslash A_2)$

\subsubsection{$(A_1\backslash A_2) \Delta (B_1 \backslash B_2) \subset (A_1\Delta A) \cup (B_1 \Delta B_2)$}
if $x \in (A_1 \backslash (A_2 \cup B_1))$ then $x \in (A_1 \ A_2)$\\
if $x \in (A_1 \backslash (A_2 \cup B_2 ^ c))$ then $x \in (A_1 \ A_2)$
if $x \in (B_1 \backslash (B_2 \cup A_1))$ then $x \in (B_1 \ B_2)$
if $x \in (B_1 \backslash (B_2 \cup A_2 ^ c))$ then $x \in (B_1 \ B_2)$

\subsection{ Let S be a semi-ring. Prove that ring generated by S is the set of all finite unions of element of S ${A \in \mathfrank{P}(X) , A=S_1\cup \dots \cup S_n \in S}$}
\begin{proof}

Let A be the set of all the finite union of elements of S, take $X,Y \in A$ then $X=\{S_i\}_{i \in \alpha}$ , $Y=\{S_i\}_{i \in \beta}$ where $\alpha$ and $\beta$ are some finite index set and $S_i \in S$    
\begin{center}
 $$ 1)  XuY=\cup_{i \in \alpha} S_i  \cup \cup_{i \in \beta} S_i = \cup _{i \in \alpha \cup \beta} S_i \in A$$
   $$ 2) X\cap Y= \cup_{i \in \alpha} S_i  \cap \cup_{i \in \beta} S_i= \cup_{i \in \alpha \& j \in \beta} (S_i \cup S_j) 
    \text{ but } S_i,S_j \in S $$  $$\text{ and S is semi ring so } S_i \cup S_j \in S. Hence
    \cup_{i \in \alpha \& j \in \beta} (S_i \cup S_j) \in A   $$
     
    $$3) X \backslash Y= \cup_{i \in \alpha} S_i  \backslash \cup_{i \in \beta} S_i= \cup_{i \in \alpha} S_i  \cap  \cap_{i \in \beta} S_i^x$$
    $$=\cup_{i \in \alpha \& j \in \beta} S_i \backslash S_j \text{ but } S_i,S_j \in S \text{ and S is a semi-ring  so } S_i \backslash S_j = \cup_{j \in \alpha_{i,j}} D_j \text{ where } \alpha_{i,j} \text{ is a index set so} $$
   $$ X\backslash y=\cup_{i \in \alpha \& j \in \beta} \cup_{l \in \alpha_{i,j}} D_l$$

\end{center}
So A is a ring generated by semi ring S.

Now suppose there is a ring R generated by S but does not contain every finite union of the elements of S. But $S \subset R$ , and R is a ring so it is closed under union hence every finite union of elements of S has to be in R $\Rightarrow\!\Leftarrow$. Thus the minimal Ring generated by S is ${A \in \mathfrank{P}(X) , A=S_1\cup \dots \cup S_n \in S}$ 
\end{proof}

\subsection{ Let $X=\{a,b,c\}$ describe all rings and all algebras in the power set of X}
The Algerbras are:\\
$$\{ \emptyset, \{a\},\{b,c\},X\}, \{ \emptyset ,\{b\},\{a,c\},X\},\{\emptyset,\{c\},\{a,b\},X\}$$
$$\{\emptyset,\{a\}\{b\},\{c\},X\}$$
$$\{\emptyset,X\}$$

The rings are:\\
$$\{\emptyset,\{a\}\},\{\emptyset,\{b\}\},\{\emptyset,\{c\}\}$$
$$\{\emptyset,\{a,b\}\},\{\emptyset,\{a,c\}\},\{\emptyset,\{b,c\}\}$$
$$\{\emptyset,\{a\},\{b\},\{a,b\}\},\{\emptyset,\{a\},\{c\},\{a,c\}\},\{\emptyset,\{b\},\{c\},\{b,c\}\}$$

\subsection{Let $R_1 \subset \mathfrank{P}(X_1)$ and $R_2 \subset \mathfrank{P}(X_2)$ be rings show that $\{A \times B \subset \mathfrank{P}(X_1 \times X_2) | A \in R_1, B \in R_2\}$ is not necessarily a ring}
Let $A=B=\{\emptyset,\{a\},\{b\},\{a,b\}\}$ the $A \times B=\{\emptyset \times \{a\},\emptyset \times \{b\},\emptyset \times \{a,b\},\{a\}\times \{b\},\{a\}\times \{a,b\},\{b\}\times \{a,b\},\{a,b\}\times \{a,b\}\}$ clearly $\{a\}\times \{b\} \cup \{a\}\times \{a,b\} \not \in A \times B$

\subsection{. Show that the union of algebras is not necessarily an algebra. What about the union of an increasing collection of algebras?}
 Let $S_1= \{\emptyset,\{a\},\{b\},\{a,b\}\}$ and $S_2=\{\emptyset,\{b\},\{c\},\{b,c\}\}$ then $S_1 \cup S_2= \{\emptyset,\{a\},\{b\}\{c\},\{b,c\},\{a,b\}\}$ clearly this set is not closed under union bc $\{b,c\}\cup \{a,b\} =\{a,b,c\} \not \in S_1 \cup S_2$
\\
Also union of infinite Algebra is not an algebra either. Let $A_n =\{X| X \subset [0,n] \cap \mathbb{N}\}$ then each $A_n$ is an algebra. However $A=\cup_{n=1} ^{\infty} A_n$ 
is not algebra bc.
take x_i st $x_i= \cup_{j=1} ^ i  ([0,j]\cap \mathbb{N})$ then all of $x_i$s do not belong to any of the $A_n$s so $\cup_{i=1} ^{\infty} \{x_i\} \not \in A $.


\subsection{Is the union of increasing collection of $\sigma$ algebras a $\sigma$ algebra.}
No the example from previous question should work \\

\begin{tcolorbox}
Claim $A_n$ is $\sigma$algebra:
$A_n$ is clearly closed under finite union and intersection and also subtraction. For countable union every $A_n$ only has finite element so the countable union would reduce to finite union by pigeon hole principle bc we should be union at least one of the element of $A_i$ infinitely many time. 
\end{tcolorbox}
\subsection{Let A be an algebra of subsets of X and $B, C \not \in A$. Is it true that $C \cup B$ is not in A?}
Let $X=\{a,b,c\}$ then $A=\{\emptyset,\{a,b\},\{c\},\{a,b,c\}\}$ is an algebra but $\{b,c\},\{a\} \not \in A$ but $\{b,c\} \cup \{a\}  \in A$ 
\subsection{Show that rectangles of the form $[a1, b1) \times [a2, b2) \subset \mathbb{R}^2$ form a semi-ring}
Claim 1) $[a1, b1) \times [a2, b2) \cap [c1, d1) \times [c2, d2)= [a1,b1) \cap [c1,d1) \times [a2,b2) \cap [c2,d2)$ Let $x=(x_1,x_2) \in [a1, b1) \times [a2, b2) \cap [c1, d1) \times [c2, d2)$ iff $x_1 \in [a1,b1)$ and  $x_1 \in [c1,d1)$ similarly $x_2 \in [a2,b2)$ and $x_2 \in [c2,d2)$ if $x_2 \in [a1, b1) \times [a2, b2) \cap [c1, d1) \times [c2, d2) = [a1,b1) \cap [c1,d1) \times [a2,b2) \cap [c2,d2)$\\
 Claim 2)Let $[a1, b1) \times [a2, b2) \cap [c1, d1) \times [c2, d2)=[a,b) \times[c,d)$
 $[a1, b1) \times[a2, b2) \backslashx [c1, d1) \times [c2, d2)=[a1,c1)\times[a2,c2) \cup [c1,b1)\times [a2,c2)\cup [a1,c1) \times [c2,b2)$
 \subsection{Let $M(A)=(b1-a1)(b2-a2)$ be the area of rectangle $A = [a1, b1) \times [a2, b2)$. Show that m is $\sigma$-additive on the semi-ring of rectangles.}
 Lets $P_x,P_y$ projection of a rectangle along the axis. Then if the beginning point and the end point of the projection is not dense in $\mathbb{R}$ along an axis we claim that rectangle are sigma additive. 
 \begin{proof}
 suppose the beginning point and the end point of the projection is not dense along X axis. Then the beginning point and the end point of the projections of squares imposes a grid on the rectangle with countably many squares. I claim along each 
 \end{proof}
\section{Book-problems}
\subsection*{problem 3}
\newpage
\subsection*{problem 4}

let $f_n=\[ \begin{cases} 
      2^{-(2n+2)}.x - \frac{1}{2^{(2n-1)}}-\frac{1}{2^{(2n+3)}}& 0 \leq x \leq \frac{1}{2^{n+1}}\\
      2^{-(2n+2)}.x+ \frac{1}{2^{(2n-1)}}+\frac{1}{2^{(2n+3)}} & 1 \geq x \geq \frac{1}{2^{n+1}}
   \end{cases}
\]$
\\
then $\int_0 ^1 f_n(x) = 1$ for all n. but $\lim_{n \rightarrow \infty} f_n = \infty$
for $x=\frac{1}{2}$ so $lim_{n \rightarrow \infty}\int _0 ^1 (f_n(x))dx =1$ while $\int _0 ^1 lim_{n \rightarrow \infty}(f_n(x))dx $ diverges 
\subsection*{problem 5}
Define $f_m(x) = \lim_{n \rightarrow \infty} (cos(m!\pi x))^{2n}$ and let $f =\lim_{m \rightarrow \infty} f_m(x)$. let $x \not \in \mathbb{Q}$ then $m!x \not \in  \mathbb{Z}$  $\forall m \in \mathbb{N}$ so $|cos m!x\pi| < 1$ hence $\lim_{n \rightarrow \infty} (cos(m!\pi x))^{2n}=0$ \\
if $x \in \mathbb{Q}$ say $x=p/q$ and if $m>q$ then $m!x$ is an integer so $f(x)=1$
\end{document}  
          